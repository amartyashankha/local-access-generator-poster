\begin{block}{Bucketing Approach \& \emph{Random-Neighbor} Queries}

\begin{itemize}
    %\item [] \textbf{Bounding the number of re-samplings:}
    \item Divide each row of the adjacency matrix into contiguous buckets
    \item Expected number of neighbors in a bucket is $\Theta(\log n)$
    \item Each vertex $v$ is associated with buckets $ \langle B^v_1, B^v_2, B^v_3,\cdots\rangle$
    \item An \textbf{unfilled} bucket may contain some indirectly exposed neighbors
    \item A \textbf{filled} bucket will contain every possible sampled neighbor
\end{itemize}


\setbeamercolor{block alerted title}{bg=Periwinkle} % Change the alert block title colors
\begin{alertblock}{Filling the $i^{th}$ bucket $B^v_i$ of vertex $v$}
\begin{itemize}
    \item Use skip-sampling to produce a \textbf{potential} \emph{next-neighbor} $u$ of $v$ in $B^v_i$
    \item Check if $(u, v)$ was set to $0$, by looking at bucket $\mathcal B$ of $u$ containing $v$
    \item If so, re-sample. Otherwise, mark $u$ as a neighbor of $v$, and update $\mathcal B$
    \item W.h.p, only $\mathcal O(\log^2 n)$ \textbf{potential} neighbors are generated in $B^v_i$
    %\item With high probability, none of the buckets contain zero neighbors.
\end{itemize}
\end{alertblock}

\end{block}



\vspace{-0.2in}
%\begin{columns}[t,totalwidth=\twocolwid]

%\begin{column}{0.49\twocolwid}

\setbeamercolor{block alerted title}{bg=Periwinkle} % Change the alert block title colors
\begin{alertblock}{Random-Neighbor($v$)}
\begin{itemize}
    \item Choose a random bucket $\mathcal B$ of $v$. If the $\mathcal B$ is \textbf{unfilled}, fill it.
    \item If $k$ neighbors found in $\mathcal B$, start over (reject) with probability $1-k/M$.
    \item If accepted, return an uniformly random neighbor found in $\mathcal B$.
\end{itemize}
For $M = \mathcal O(\log^2 N)$, the max number of neighbors in any bucket is $<M$.
So, the number of rejection sampling rounds is $\mathcal O(\log^2 N)$ in expectation.
\end{alertblock}

%\end{column}


%\begin{column}{0.49\twocolwid}

%\begin{alertblock}{Degree Sampling}
%\end{alertblock}

%\end{column}

%\end{columns}



\begin{block}{Stochastic Block Model}%: Random Community Assignment}

\begin{itemize}
    \item Each vertex is assigned to some community $C_i\subseteq V$ for $i\in [r]$
    \item Communities $\{C_i\}_{i\in [r]}$ partition $V$:
          if $u\in C_i, v\in C_j$, then $\mathbb P_{(u, v)\in E} = p_{ij}$
    %where $\left\{ p_{ij}\right\}_{i,j\in [r]}\in [0,1]^{r\times r}$
\end{itemize}
\begin{itemize}
\item [] \textbf{Given sizes of each comunity $C_i$ and a range of length $\ell$}
    \item Count number of occurrences of each community in any contiguous range%: $|[a,b]\cap C_j|$
    \item Sample from \emph{Multivariate Hypergeometric Distribution}%: $O(r\, poly(\log n))$ resources
\[
\Pr[\mathbf{S}^\mathbf{C}_\ell = \langle s_1, \ldots, s_r \rangle]
= \frac{\binom{C_1}{s_1}\cdot\binom{C_2}{s_2}\cdots\binom{C_r}{s_r}}{\binom{B}{\ell}}
\hspace{2ex}\textrm{ where } \ell = \mathlarger{\sum}^{r}_{i=1} s_i
\textrm{ and } B = \mathlarger{\sum}^{r}_{i=1} C_i
\]
\end{itemize}


\setbeamercolor{block alerted title}{bg=CadetBlue} % Change the alert block title colors
\begin{alertblock}{Multivariate Hypergeometric Distribution}
[cite] solves the special case of $r=2$ and $B = 2\ell$.
\begin{itemize}
    \item \textbf{Extending to $B\not= 2\ell$}: Divide $\ell$ into $\mathcal O(\log n)$ dyadic segments.
    \item \textbf{Extending to $r>2$}: Make a tree with $r$ leaves (one for each $C_i$).
          Every branch down the tree is equivalent to a $r=2$ splitting.
\end{itemize}
\end{alertblock}

\end{block}
