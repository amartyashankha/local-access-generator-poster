
\documentclass[final]{beamer}

\usepackage[scale=1.35,size=custom,width=42,height=48]{beamerposter} % Use the beamerposter package for laying out the poster

\usepackage{amsthm}
\usepackage{amsmath}
\usepackage{transparent}
\usepackage{relsize}
\usepackage{lmodern}
\usepackage{exscale}
\usepackage{anyfontsize}
\usepackage{pdfpages}
\usepackage{fontspec}
%\usepackage{parskip}


\usepackage{graphicx, import}  % Required for including images

\usetheme{confposter} % Use the confposter theme supplied with this template

\setbeamercolor{block title}{fg=ngreen,bg=white} % Colors of the block titles
\setbeamercolor{block body}{fg=black,bg=white} % Colors of the body of blocks
\setbeamercolor{block alerted title}{fg=white,bg=dblue!70} % Colors of the highlighted block titles
\setbeamercolor{block alerted body}{fg=black,bg=dblue!10} % Colors of the body of highlighted blocks
% Many more colors are available for use in beamerthemeconfposter.sty

%-----------------------------------------------------------
% Define the column widths and overall poster size
% To set effective sepwid, onecolwid and twocolwid values, first choose how many columns you want and how much separation you want between columns
% In this template, the separation width chosen is 0.024 of the paper width and a 4-column layout
% onecolwid should therefore be (1-(# of columns+1)*sepwid)/# of columns e.g. (1-(4+1)*0.024)/4 = 0.22
% Set twocolwid to be (2*onecolwid)+sepwid = 0.464
% Set threecolwid to be (3*onecolwid)+2*sepwid = 0.708

\newlength{\sepwid}
\newlength{\leftcolwid}
\newlength{\rightcolwid}
\newlength{\twocolwid}
\setlength{\paperwidth}{48in} % A0 width: 46.8in
\setlength{\paperheight}{42in} % A0 height: 33.1in
\setlength{\sepwid}{0.0125\paperwidth} % Separation width (white space) between columns
\setlength{\leftcolwid}{0.285\paperwidth} % Width of one column
\setlength{\twocolwid}{0.38\paperwidth} % Width of two columns
\setlength{\rightcolwid}{0.285\paperwidth} % Width of one column
\setlength{\topmargin}{-0.5in} % Reduce the top margin size

\usepackage{xcolor}
\usepackage{soul}

\newcommand{\hlc}[1]{{%
    \colorbox{yellow}{#1}}%
}

\renewcommand{\baselinestretch}{1.05}
\setbeamertemplate{itemize item}{$\scriptstyle\blacktriangleright$}

%\setmainfont[Ligatures=TeX]{Cambria}

%-----------------------------------------------------------


%----------------------------------------------------------------------------------------
%	TITLE SECTION 
%----------------------------------------------------------------------------------------

\title{Local-Access Generators} % Poster title

\author{Amartya Shankha Biswas, Ronitt Rubinfeld, Anak Yodpinyanee} % Author(s)

\institute{CSAIL, MIT} % Institution(s)

%----------------------------------------------------------------------------------------

\begin{document}


\addtobeamertemplate{block end}{}{\vspace*{2ex}} % White space under blocks
\addtobeamertemplate{block alerted end}{}{\vspace*{2ex}} % White space under highlighted (alert) blocks

\setlength{\belowcaptionskip}{2ex} % White space under figures
\setlength\belowdisplayshortskip{2ex} % White space under equations

\begin{frame}[t] % The whole poster is enclosed in one beamer frame


\begin{columns}[t] % The whole poster consists of three major columns, the second of which is split into two columns twice - the [t] option aligns each column's content to the top

\begin{column}{\sepwid}\end{column} % Empty spacer column

\begin{column}{\leftcolwid} % The first column

\setbeamercolor{block title}{fg=Mahogany,bg=white} % Change the block title color
\begin{block}{Partial Sampling from a Distribution}



\setbeamercolor{block alerted title}{bg=Emerald} % Change the alert block title colors
\begin{alertblock}{Full Sampling $R \sim\mathsf D$ in $\mathcal O (T)$ time}

\begin{figure}[h!]\centering
    \def\svgwidth{0.5\columnwidth}
    \import{svg/}{generic_sampling.pdf_tex}
\end{figure}
Do we need to spend $\mathcal{O}(T)$ upfront?

\end{alertblock}


\begin{alertblock}{$N$ steps of \emph{Partial Sampling}}

\begin{center}
Each partial step should take $\tilde{\mathcal O} (T/N)$ time.
\end{center}

\begin{figure}[h!]\centering
    \def\svgwidth{0.7\columnwidth}
    \import{svg/}{partial_sampling.pdf_tex}
\end{figure}

\end{alertblock}


\setbeamercolor{block alerted title}{bg=Bittersweet} % Change the alert block title colors
\begin{alertblock}{\textbf{Problem Statement}}

A local-access generator of a random object $R \sim\mathsf D$,
provides indirect access to $R'$ with a \emph{query oracle} s.t.
\begin{itemize}
    \item All query responses (\emph{partial samples}) are \textbf{consistent}
    \item The \textbf{distribution} of $R'$ is $\epsilon$-close to $\mathsf D$ in $L_1$ distance
\end{itemize}

\end{alertblock}



\end{block}


\begin{block}{Sampling $G(n, p)$: \textsf{Vertex-Pair} queries}

%\textbf{Model:} $n$-vertex undirected graph: edge probability $p$

\textbf{\textsf{Vertex-Pair}}: Given vertices $u, v$, decide whether $(u,v)\in E$.
%\begin{itemize}
%    \item [] \textbf{Query Model:} Given vertices $u, v$, is $(u,v)\in E$?
    %\item 
    \emph{Trivial}: just a collection of $n \choose 2$ Bernoulli RVs with bias $p$.
%\end{itemize}

\end{block}

\begin{block}{\textsf{Next-Neighbor} queries (skip-sampling)}

\textbf{\textsf{Next-Neighbor}}: Return neighbors of $v$ in order.

\vspace{15pt}

\emph{\color{red}Issue:} Na\"ive sampling spends $1/p$ time sampling the $0$'s

\emph{Idea}: can compute \textsf{Next-Neighbor}'s distribution's CDF from
\vspace{-10pt}
\[ \mathbb P[k \textrm{ non-neighbors before next-neighbor}] = p(1-p)^k \]

\colorbox{BlueGreen}{\textbf{Skip-sampling}} Draw from \textsf{Next-Neighbor} distribution
\vspace{-20pt}
\begin{itemize}
    \item Can sample from this distribution in $\tilde{\mathcal O}(1)$ time [ELMR17]
    \item Further analysis required for finite-precision arithmetic
%    \item Na\"ive sampling spends $1/p$ time sampling the $0$'s
\end{itemize}

\emph{\color{red}Issue:} Adjacency matrix is symmetric; we need to \textbf{record all generated $0$'s} in the corresponding column of $v$

\begin{figure}[h!]\centering
    \def\svgwidth{0.9\columnwidth}
    \import{svg/}{skip_sampling.pdf_tex}
\end{figure}
\emph{\color{red}Issue:} if the sampled neighbor is already $0$, must re-sample

\quad$\Rightarrow$ may hit $0$'s many times -- \textbf{too many re-samplings}

\end{block}


\end{column} % End of the first column


\begin{column}{\sepwid}\end{column} % Empty spacer column


\begin{column}{\twocolwid} % Begin a column which is two columns wide (column 2)


\setbeamercolor{block title}{fg=Mahogany,bg=white} % Change the block title color
\begin{block}{Bucketing Approach \& \emph{Random-Neighbor} Queries}

\begin{itemize}
    %\item [] \textbf{Bounding the number of re-samplings:}
    \item Divide each row of the adjacency matrix into contiguous buckets
    \item Expected number of neighbors in a bucket is $\Theta(\log n)$
    \item Each vertex $v$ is associated with buckets $ \langle B^v_1, B^v_2, B^v_3,\cdots\rangle$
    \item An \textbf{unfilled} bucket may contain some indirectly exposed neighbors
    \item A \textbf{filled} bucket will contain every possible sampled neighbor
\end{itemize}


\begin{alertblock}{Filling the $i^{th}$ bucket $B^v_i$ of vertex $v$}
\begin{itemize}
    \item Use skip-sampling to produce a \textbf{potential} \emph{next-neighbor} $u$ of $v$ in $B^v_i$
    \item Check if $(u, v)$ was set to $0$, by looking at bucket $\mathcal B$ of $u$ containing $v$
    \item If so, re-sample. Otherwise, mark $u$ as a neighbor of $v$, and update $\mathcal B$
    \item W.h.p, only $\mathcal O(\log^2 n)$ \textbf{potential} neighbors are generated in $B^v_i$
    %\item With high probability, none of the buckets contain zero neighbors.
\end{itemize}
\end{alertblock}

\end{block}



\vspace{-0.2in}
%\begin{columns}[t,totalwidth=\twocolwid]

%\begin{column}{0.49\twocolwid}

\begin{alertblock}{Random-Neighbor($v$)}
\begin{itemize}
    \item Choose a random bucket $\mathcal B$ of $v$. If the $\mathcal B$ is \textbf{unfilled}, fill it.
    \item If $k$ neighbors found in $\mathcal B$, start over (reject) with probability $1-k/M$.
    \item If accepted, return an uniformly random neighbor found in $\mathcal B$.
\end{itemize}
For $M = \mathcal O(\log^2 N)$, the max number of neighbors in any bucket is $<M$.
So, the number of rejection sampling rounds is $\mathcal O(\log^2 N)$ in expectation.
\end{alertblock}

%\end{column}


%\begin{column}{0.49\twocolwid}

%\begin{alertblock}{Degree Sampling}
%\end{alertblock}

%\end{column}

%\end{columns}



\begin{block}{Stochastic Block Model}%: Random Community Assignment}

\begin{itemize}
    \item Each vertex is assigned to some community $C_i\subseteq V$ for $i\in [r]$
    \item Communities $\{C_i\}_{i\in [r]}$ partition $V$:
          if $u\in C_i, v\in C_j$, then $\mathbb P_{(u, v)\in E} = p_{ij}$
    %where $\left\{ p_{ij}\right\}_{i,j\in [r]}\in [0,1]^{r\times r}$
\end{itemize}
\begin{itemize}
\item [] \textbf{Given sizes of each comunity $C_i$ and a range of length $\ell$}
    \item Count number of occurrences of each community in any contiguous range%: $|[a,b]\cap C_j|$
    \item Sample from \emph{Multivariate Hypergeometric Distribution}%: $O(r\, poly(\log n))$ resources
\[
\Pr[\mathbf{S}^\mathbf{C}_\ell = \langle s_1, \ldots, s_r \rangle]
= \frac{\binom{C_1}{s_1}\cdot\binom{C_2}{s_2}\cdots\binom{C_r}{s_r}}{\binom{B}{\ell}}
\hspace{2ex}\textrm{ where } \ell = \mathlarger{\sum}^{r}_{i=1} s_i
\textrm{ and } B = \mathlarger{\sum}^{r}_{i=1} C_i
\]
\end{itemize}


\begin{alertblock}{Sampling the Multivariate Hypergeometric Distribution}
[cite] solves the special case of $r=2$ and $B = 2\ell$.
\begin{itemize}
    \item \textbf{Extending to $B\not= 2\ell$}: Divide $\ell$ into $\mathcal O(\log n)$ dyadic segments.
    \item \textbf{Extending to $r>2$}: Make a tree with $r$ leaves (one for each $C_i$).
          Every branch down the tree is equivalent to a $r=2$ splitting.
\end{itemize}
\end{alertblock}

\end{block}

\begin{block}{Stochastic Block Model}%: Random Community Assignment}

%\begin{itemize}
    %\item Each vertex is assigned to some community $C_i\subseteq V$ for $i\in [r]$
Communities $\{C_i\}_{i\in [r]}$ partition $V$: If $u\in C_i, v\in C_j$, then $\mathbb P_{(u, v)\in E} = p_{ij}$.
    %where $\left\{ p_{ij}\right\}_{i,j\in [r]}\in [0,1]^{r\times r}$
%\end{itemize}
\begin{itemize}
\item [] \textbf{Given sizes of each comunity $C_i$ and a range of length $\ell$}
    \item Count number of occurrences of each community in any contiguous range%: $|[a,b]\cap C_j|$
    \item Sample from \emph{Multivariate Hypergeometric Distribution}%: $O(r\, poly(\log n))$ resources
\[
\Pr[\mathbf{S}^\mathbf{C}_\ell = \langle s_1, \ldots, s_r \rangle]
= \frac{\binom{C_1}{s_1}\cdot\binom{C_2}{s_2}\cdots\binom{C_r}{s_r}}{\binom{B}{\ell}}
\hspace{2ex}\textrm{ where } \scriptstyle{
    \ell = \mathlarger{\sum}\limits^{r}_{i=1} s_i \textrm{ and } B = \mathlarger{\sum}\limits^{r}_{i=1} C_i
}
\]
\end{itemize}

\begin{figure}[h!]\centering
    \def\svgwidth{0.9\columnwidth}
    \import{svg/}{counting.pdf_tex}
\end{figure}


\end{block}



\end{column} % End of the second column



\begin{column}{\sepwid}\end{column} % Empty spacer column



\begin{column}{\rightcolwid} % The third column

\setbeamercolor{block alerted title}{bg=CadetBlue} % Change the alert block title colors
\begin{alertblock}{Multivariate Hypergeometric Distribution}
[GGN10] solves the special case of $r=2$ and $B = 2\ell$.
\begin{itemize}
    \item [] $\textsc{Counting-Generator}$
    \item \textbf{Extending to $B\not= 2\ell$}: Divide $\ell$ into dyadic segments.
    \item \textbf{Extending to $r>2$}: Make a tree with a leaf for each $C_i$
          Every branch in the tree is equivalent to a $2$-splitting
\end{itemize}
\end{alertblock}

\vspace{-25pt}
\begin{itemize}
    \item Use $\textsc{Counting-Generator}$ to sample community counts
    \item Run the $\textsc{Bucketing-Generator}$ as before
\end{itemize}



\setbeamercolor{block title}{fg=Mahogany,bg=white} % Change the block title color
\begin{block}{Work in Progress}



\setbeamercolor{block alerted title}{bg=norange} % Change the alert block title colors
\begin{alertblock}{Random Domino Tiling}
A $2\times n$ grid tiled with dominoes: $F_n$ tilings possible.
\begin{figure}[h!]\centering
    \def\svgwidth{0.8\columnwidth}
    \import{svg/}{domino_tiling.pdf_tex}
\end{figure}
\textbf{Query:} $i^\textrm{th}$ column configuration: \emph{vertical}, \emph{left}, \emph{right}?
\begin{figure}[h!]\centering
    \def\svgwidth{0.8\columnwidth}
    \import{svg/}{domino_partial.pdf_tex}
\end{figure}
Sufficent to approximate $F_c/F_{c-1}$: Use $\phi$ if $c = \Omega(\log(n))$
\textbf{Open:} $k\times n$ grid for $k = \omega(1)$ and Dimer model
\end{alertblock}


\begin{alertblock}{Random Coloring: Glauber Dynamics}
Consider a \emph{uniform} random coloring of the input graph.
\textbf{Query:} What is $v$'s color in the random coloring?
%Find random $k$-coloring for graph with max degree $\Delta$
\begin{itemize}
    \item [] \underline{\textbf{Global Algorithm}} (Glauber Dynamics) for $k > 2\Delta$
    \item Sample $r=\mathcal O(n\log n)$ (vertex, color) pairs $\langle(v_i, c_i)\rangle_{i\in[r]}$
%          $\left\{ (v_1, c_1), (v_2, c_2), (v_3, c_3), \cdots, (v_r, c_r)\right\}$
    \item For steps $i = 1, \ldots, r$
    \begin{itemize}
        \item If no neighbor of $v_i$ has color $c_i$, set $v_i$'s color to $c_i$
        \item Else, do nothing
    \end{itemize}
\end{itemize}

\begin{itemize}
    \item [] \underline{\textbf{Local Algorithm}} for $k = \Omega(\Delta\log n)$
%    \item Given $v$, what is $v$'s color (in some random coloring)?
    \item Locally sample \emph{all} occurences of $(v, \star)$: implemented effenciently with the proposed \textbf{Counting-Generator}
    \item Sample $(w, \star)$ if necessary, where $w$ is neighbor of $v$
    \item Query tree is of size $\mathcal{O}(1)$ for $k = \Omega(\Delta\log n)$
\end{itemize}
\textbf{Open:} $k = o(\Delta\log n)$
\end{alertblock}




\end{block}



\end{column} % End of the third column


\end{columns} % End of all the columns in the poster




\begin{column}{0.7\paperwidth}

\let\thefootnote\relax\footnotetext{
    \scriptsize{
        [ELMR17] Guy Even, Reut Levi, Moti Medina, and Adi Rosen.
        Sublinear random access generators for preferential attachment graphs.
        In 44th International Colloquium on Automata, Languages, and Programming,
        ICALP 2017, July 10-14, 2017, Warsaw, Poland, pages 6:1–6:15, 2017.%, pages 6:1-6:15.
    }
}
\let\thefootnote\relax\footnotetext{
    \scriptsize{
        [GGN10] Oded Goldreich, Shafi Goldwasser, and Asaf Nussboim.
        On the implementation of huge random objects.
        SIAM Journal on Computing, 39(7):2761–2822, 2010.
    }
}

\end{column}



\end{frame} % End of the enclosing frame





\end{document}
